% This template was provided for the 2019 Recording Project class at UWL.
% It is designed to provide a simple, low-configuration starter template
% for academic writing with LaTeX and Markdown.
% 
% This template is set up to use the agsm Harvard referencing style by default.

\documentclass[a4paper]{scrartcl}

\usepackage[english]{babel}
\usepackage[utf8]{inputenc}
\usepackage{amsmath}
\usepackage{graphicx}
\usepackage[T1]{fontenc}
\usepackage[utf8]{inputenc}
\usepackage{lmodern}
\usepackage[english]{babel}
\usepackage{csquotes}
\usepackage{natbib}
\usepackage[footnotes,definitionLists,hashEnumerators,smartEllipses,hybrid,citations]{markdown}
\usepackage{longtable}

\newcommand{\hash}{\#} % Hashes are difficult in the Markdown environment. typing \hash within a Markdown document will give you a working hash instead.
\setlength{\parskip}{1em}

% A basic Harvard style is the agsm style. Other similar styles are apalike and lsalike.

\bibliographystyle{agsm}
\setkeys{Gin}{width=.95\linewidth}
\pagenumbering{gobble}% Remove page numbers (and reset to 1)

% Add your report title here:
\title{Markdown, LaTeX, and Harvard Referencing starter template}

% Add your name here:
\author{Eddie W.}

% The command below auto-generates today's date.
\date{\today}

% The \begin{} command tells LaTeX that we're starting the document.
\begin{document}

% \maketitle prints the title (as defined in the previous section) to the page.
\maketitle

% \newpage is fairly self-explanatory!
\newpage

% \tableofcontents reads the whole document and auto-generates a table of contents based on your section headings.
\tableofcontents

% \listoffigures generates a list of images used in the document.
\listoffigures

\newpage
\pagenumbering{arabic}% Arabic page numbers (and reset to 1)

% This is where your work goes!
% The \markdownInput{} command inserts the contents of a Markdown file into the document body. The first file 
\markdownInput{01-introduction.md}

% This command inserts a different Markdown file into the main document. Using multiple files this way allows you concentrate on single sections at a time. Of course, it's perfectly fine to write your entire report in a single Markdown file if you would prefer.
\markdownInput{02-second-section.md}

\markdownInput{03-images.md}

\markdownInput{04-citations.md}

\markdownInput{05-conclusion.md}

\newpage
\appendix
% \bibliographystyle{plainnat}
\bibliography{references}

\end{document}
